\chapter{Introduction}
% TODO Something similar to the report's abstract, first summarizing what we intend to solve, and then explain what this chapter is about.
Since IBM mainframes systems era, job schedulers\footnote{Should not be confused with process scheduling, which is 
the assignment of currently running processes to CPUs by the operating system.} are an important part of the IT infrastructure.
They are in charge of running background and unattended executions, and are typically used for system maintenance and administration jobs 
such as hard drive defragmentation, system updates check or system clock synchronization. However, they are also used by final system 
users to manage their jobs, like reminders or resource intensive processes.\\
Job schedulers must decide which job to run and when. There are some schemes or parameters that can be taken into account for taking
these decisions\cite{wp:js}.\\
\begin{list}{-}{Some of the most used are:}
  \item Defined execution time
  \item Elapsed execution time
  \item Execution time given to user
  \item Job priority
  \item Compute resource availability
  \item Number of simultaneous jobs allowed for a user
  \item Availability of peripheral devices
  \item Occurrence of prescribed events
\end{list}
\emph{cron}, probably the most popular job scheduler in the UNIX world, only considers the first parameter from the list, which is 
probably the simpler to use and the most functional. According to Wikipedia\cite{wp:cron}:
\begin{quote}
  \emph{
    ``cron is a time-based job scheduler in Unix-like computer operating systems. cron 
    enables users to schedule jobs (commands or shell scripts) to run periodically at 
    certain times or dates. It is commonly used to automate system maintenance or 
    administration, though its general-purpose nature means that it can be used for other 
    purposes, such as connecting to the Internet and downloading email.''
  }
\end{quote}
As we can read in this entry, \emph{cron} is claimed to be a general-purpose job scheduler. But we can be more accurate and say that it is 
general-purpose in terms of 'action', but not in terms of 'reaction'. In other words, \emph{cron} can execute any action, but as we 
said before, it only considers the defined execution time to do so, which we could refer to it as just one reaction parameter.\\
There are more popular services with job-scheduling capabilities based on other parameters like \emph{udev}, which is a device manager that 
can run shell commands when the availability of some peripheral device changes. Or \emph{syslog}, that can also run shell commands when 
it receives a log message from an application. Receiving a log message is not a listed scheduling parameter, but it is easy to notice that
the message can be considered an event. In fact, most of those listed scheduling schemes can be defined as occurrences of prescribed 
events. The rest of them are states that need to be looked up.\\
Our software project, called \emph{reactor} so far, is a general-purpose job scheduler and event handler. Its main goal is to be 
general-purpose on both 'action' and 'reaction' ways.
\\
By now we have set out some basic concepts about job-scheduling that will be present for the rest of this final report. On the next 
sections to come we are going to expose a use case of this software project. We will also make a brief statement about the goals to 
achieve according to this use case, and explain the first decisions made to develop the solution.\\
\emph{reactor} is intended to survive this final project, so there will be forecast goals that would not be implemented at the time this 
document is released. All this will be detailed.\\
As a final remark I want to confirm that English is not my mother tongue, but we found it useful to write this document in English so it
could be useful to a wider range of people, if any.

% \section{Final report contents summary}
% TODO Explain the chapters on blocks.

\section{Motivation}
\label{moti}
% TODO This is not a mandatory section, but it would be good to explain how the idea came in mind and why do we want 
%   to implement it as a final project.
In the last years a big wave of mobile devices with integrated GPS sensors came in to stay. We are talking about smartphones, tablets,
digital cameras... Handy and very portable devices. But nowadays you can also get geolocation information from not-so-portable devices
like a laptop or a desktop computer. You can obviously connect your car's GPS to the computer, and if the controller allows it, you will 
be able to do exactly the same as with an smartphone, but this is not a usual use case. A most common situation is to be connected to the
internet and expect location-related results from a web search, without extra peripherals and hassles. This is something that work
thanks to Wi-Fi and ISP IP geolocation databases.\\
So we have a bunch of devices able to geolocate themselves. What can we do with such a feature? A lot of things have been done, but the
major part of them seem to be a functionalities for the services providers more than for the users, for example showing location-related
advertisements or creepy user tracking. We can make a good use of it by making a geolocation-based job scheduler, so the devices 
geolocation would be the reaction parameter for our job scheduler. For instance we could automatically set our smartphone in silent mode 
when we are at the theatre. This was the first idea that came in mind about a good personal project to develop. A more or less simple 
job scheduler daemon running along with cron if not with more job schedulers, without any communication between them. This was not 
enough, so the idea of the project began to become bigger and bigger, and so it became more abstract and modular. Also mobile devices have
more sensors than GPS, like gyroscopes and accelerometers, that could be used too. We wanted a job-scheduler with the ability of reacting 
to several parameters, which would be able to interact between them. Something like 'cron meets GPS and more'.\\
\\
Making money is usually the main motivation for a project, but this is not our case because there is no intention to sell it. The project 
will be FLOSS\footnote{Free/Libre Open-Source Software}, so the real motivations are those that usually come with this kind of projects. 
Learning from the experience of developing a long-term personal project from scratch is the major one. Then the project should be useful 
to myself as by now I am the only target interested on it. Finally it may also be useful for other people not involved in the project 
development, what is expected and highly desirable to the point that the project is developed with them always in mind.
% TODO Anything more?
\section{Use case}
\label{usecase}
Here we describe a use case for this project so we can have a reference in mind for the rest of the document to illustrate
and help us to understand it.\\
\\
For this use case we will think of a web developer who works at an office with his own laptop.\\
He begins the day at home, where with his laptop checks his personal mail account and sets some tasks and appointments to his personal 
calendar software. Also his smartphone is in 
normal mode (both ringtone and vibration are on).\\ % Here we can make reactor trigger some action from a calendar alert
Thanks to our software, when he arrives to the office the cellphone is in vibration mode. Also when he starts his computer there, the
environment has changed. Now the email notifier is not checking personal email accounts, but work accounts. The calendar shown is the
company one, and the wallpaper is more sober. It also automatically pulls the new commits from the remote server, asking for manual actions
if required, and starts his favourite IDE. Afterwards, while he is working, every error that the company's production http server logs is 
notified to him.\\
When the working day is over the default behaviour is restored. The cellphone returns to the normal mode, he receives email notifications
from his personal accounts and the personal calendar is enabled again. But is not until he leaves the office that he will stop receiving 
notifications from the server's system log and the company's email, and the company's calendar will be hidden.\\
\\
We don't have any maintenance or system administration task done by our software in this use case, and this is done in purpose. We have 
seen before that job schedulers are typically use for those kind of tasks, but they are transparent for the user. So if we put 
administration tasks on the use case, the user would not be aware of the existence of our software and would not make direct use of it.
We preferred to break with the cliché and show a tool useful not only for operating systems internals but also for the final user. Assuming
that it is useful for maintenance is just going a tiny step further from the use case.
% TODO This section must be modified while changes are needed for other sections.
\section{Brief goals description}
\label{bgoals}
The goals will be defined in detail in \emph{section \ref{sec:solution}}, after explaining the problem to solve. In this section
we want to show the main goals that can be extracted from the use case.\\
\begin{list}{-}{}
  \item Event driven
    \subitem It should react to abstract events, such as 'arrived to the office', 'error log in the http server' or 'beginning of the working 
      day'...
  \item Execution of shell commands
    \subitem All the actions described in the use case can be performed by command-line. That is what makes it general purpose in 'action'
      terms.
  \item State aware
    \subitem Notice that in the use case our web developer can receive error logs notifications from the server when he is at the office,
      but not when he is not there. So our software must be able to know when he is at work and only then react to 'error log in the http
      server'. When we are out of this state these events must be ignored. In the use case there are more examples of this, but they are
      not so clear.
  \item Communication between systems with our job-scheduler
    \subitem Also in the communication of the error log, being the http server a remote machine, one can see that there is a network 
      communication between systems.
  \item Multiple kinds of events
    \subitem The job-scheduler is little limited by the kind of events it can receive.
\end{list}

\section{Preliminary decisions}
\label{sec:preli}
Before the specification there are some decisions that were taken for a number of reasons like philosophy, ideals, learning goals... 
The fact that some of these decisions were made before we specified what we want to do may be taken as erroneous. Actually, those decisions
could be done after the specification and it wouldn't change anything, some of them even after the implementation with the same result. But
still, those decisions were personal requisites for the software project and for the final project. This is why we will also justify the 
timing and not only the decisions themselves.\\
Those decisions are related to the software license election and the platform for the project to head.
\subsection{License}
% TODO Explain the license election, talk about FLOSS software, compare some licenses, GPLv2 & GPLv3...
From the beginning there was something that was out of discussion about this project, and it was the kind of license the software project
was going to be under. As we said in a previous section, \emph{reactor} is a FLOSS project, which according to the FSF\footnote{Free 
Software Foundation is non-profit corporation that claims to have a '\emph{worldwide mission to promote computer user freedom and to defend 
the rights of all free software users}'. The activities for which they are mainly known are the GNU Project, the GNU Licenses and 
pro-FLOSS activism\cite{fsf}. For more information: \url{http://fsf.org}} is software that follows four rules or freedoms to procure.
\begin{list}{-}{These freedoms are\cite{fsf}:}
  \item The freedom to run the software, for any purpose.
  \item The freedom to study how the software works, and change it so it does your computing as you wish.
    \subitem Access to the source code is a precondition for this. 
  \item The freedom to redistribute copies so you can help your neighbour.
    \subitem By doing this you can give the whole community a chance to benefit from your changes. 
  \item The freedom to distribute copies of your modified versions to others.
\end{list}
This is not the only interpretation of what free software means, but probably is the most general and accepted one.\\
So yes, we agree with that and want it for our software. An idea is nothing more than a set of other ideas that other people had before,
so they don't belong to anybody and they can not be sealed, hidden or restricted. Otherwise it would be a childish selfish behaviour. And 
software is nothing more than a written implementation of ideas.\\
But those four freedoms are not enough for what we want. We like the giving part of the deal, but we also want to get something. If 
someone takes advantage of the second freedom, we want these changes to be public, because they are probably improvements of a project
we designed, so it is a derivative work, and we have the right to check on them and add them back to the initial project.\\
That limits the choice of the software license. Doing a quick check on the major existing licenses and their main characteristics we 
easily arrive to the conclusion that the license that fits best to our needs is the GPL. It requires the derivative work to be released
with the same terms of the license, without exceptions.\\
But there is another choice to make about the license and it is the version. The GPL has three versions. GPLv1 is the one that essentially
protects the four freedoms stated by the FSF by forcing the distributors of the software to publish the source code and license under
the same terms the modified versions, so the mix of licenses don't diminish the overall value of them. The GPLv2 adds some kind of 
protections to patent fees from software corporation to free software distributors. The GPLv2 licensed software only can be distributed 
without any condition or restriction like for example fees, if not it can not be distributed. The GPLv3 goes further on the software patent 
protection, and also states controversial clauses against the 'tivoization'. Tivoization is how the FSF calls the practice of limiting 
the execution of free software by the hardware. Its name comes from the TiVo device, which runs GPLv2 licensed software and follows the
terms of use, but it doesn't let run your modified version of the code. GPLv3 states that the software must not be restricted by the 
hardware in which it comes\cite{rms:gplv3}. This is controversial because some relevant developers think that telling how the hardware 
has to be in order to run GPLv3 software is too intrusive\cite{k:gplv2}.
We finally stand for the GPLv2 for three reasons. The first one is that the GPLv2 is stronger than GPLv1 (which is basically deprecated).
The second reason is that hardware intrusiveness arguments feels strong enough. We are making software and we don't care about the hardware
design in which it runs, it is not our work. We may prefer open hardware, but this is a personal decision, not something we want people to
be forced to. The last one is that we always can change our mind and upgrade to GPLv3 if we find it better, as it is designed to be easy 
to upgrade to.\\
There is one more special case to take into account and it is the license for the libraries, if any (and as we will see in the next 
chapters, we will have libraries). The problem of the GPL with the libraries is that the main program that links to a GPL library must be
GPL too, and that is something that we don't want. The solution is using the LGPLv2, which is very similar to the GPLv2 but allows the 
code to be linked to any program.\\
So, in conclusion, our software project will be under GPLv2 and LGPLv2.
\subsection{Platform}
Taking by platform the hardware architecture and the software framework in which our project is going to run, here we are going to show the
main platforms available, pros and cons, which one we choose and why. This will be centred on the software framework as the actual
interface we are going to deal with. But we will choose it in regard to the hardware in which it can run.\\
So we have mainly two big families of software platforms to which we can focus our project to, Microsoft Windows and POSIX operating
systems. Microsoft Windows is a widely used privative operating system on desktop and laptop form factors. More than a family, as UNIX-like
operating systems are, they are all versions of the same OS as all of them are released by the same company and they drop support for old 
versions when they launch new products. However, its API is usually quite compatible between versions. \\
In the other hand we have POSIX operating systems, which can also be called UNIX-like operating systems. POSIX is a standard API definition
for software compatibility with variants of Unix and other operating systems. So, long story short, UNIX-like operating systems are a big
bunch of operating systems running in every hardware architecture, that share almost the same API. Into that category fall the popular Mac 
OS X, iOS, Linux, Solaris and HP/UX.\\
We prefer for our software the idea of using an standard API that works in many operating systems and devices than a API widely adopted but
strictly restricted by a company and their products. That brings us a wide range of requirements to rely on. Also, we are interested on 
focus our project on Linux kernel, which is very stable and developer friendly, as well as GPLv2 licensed and very adopted on 
ultra-portable devices and mainframes. So the decision is made, our software project will be oriented to the UNIX platform.\\
There is also the possibility of making \emph{reactor} cross-platform by choosing the correct API at compilation time with tools like 
Autoconf and using cross-platform libraries. The time of this final project is limited and that makes us choose not to support Windows from 
the beginning, as it would take time to learn an API that us, as developers, are not interested into. May be in the future if there are
users interested.\\
This decision was made before the specification because we had a personal interest on learning POSIX and particularly on Linux API. Also 
the pros and cons we state before are valid for almost every project we could do.
