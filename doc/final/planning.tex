\chapter{Planning and economic study}
In this section we will check the resources estimations and the final result of this project, being those resources time and money.
\section{Tasks and temporal distribution}
\label{sec:tasks}
That is the planning followed in order to finish the project in time. Here is also where the goals of the final project are clearly 
stated. It's mainly a list of tasks to perform ordered from most to less priority, and assigned deadlines to sets of those tasks:\\
\begin{list}{-}{December 10th, 2011 ({\bf 60 days})}
  \item \emph{First specification iteration of the project}\\
    Define the main project's concept in which the rest of the project will be based, as a draft.
  \item \emph{Define use cases}\\
    Use cases to support the concept.
  \item \emph{Collect general information}\\
    Check the viability of the concept.
\end{list}
\begin{list}{-}{January 26th, 2012 ({\bf 16 days})}
  \item \emph{First design iteration of the project}\\
    After checking the viability of the project it's time to look for the better solution. This includes an implementation point of view
    for the selection of technologies.
\end{list}
\begin{list}{-}{February 2nd, 2012 ({\bf 7 days})}
  \item \emph{Makefile.am and configure.ac}\\
    Learn the basics of \emph{Automake} and \emph{Autoconf}.
  \item \emph{Log functions}
  \item \emph{'events' group users check}
\end{list}
\begin{list}{-}{February 16th, 2012 ({\bf 14 days})}
  \item \emph{Main data structure of the daemon done and working}
  \item \emph{Wrappers to all the third party data structures (mainly glib)}\\
    As it's expected to improve the efficiency of the software in the 
    future, we want to wrap all the declarations of the third 
    party basic data structures we use, so when we re-implement them
    the changes to any other non-related code will be minimal.
  \item \emph{Daemon socket ready to receive events from other processes}\\
    The poll management performed by \emph{libevent}.
  \item \emph{Execute shell commands functionality}
  \item \emph{Dummy first version of control program}\\
    Its purpose is only to send events, so we can test the daemon.
\end{list}
% TODO Move this.
After this deadline what we have are the very most basic and principal 
functionalities of the project. This is, a daemon with some states machines 
defined in it that receives event messages for those states machines from a 
socket and executes shell commands if assigned to the transitions.
There's no way to read the states machines from anywhere yet so it has some test 
hard-coded state machines.\\
\\
\begin{list}{-}{February 29th, 2012 ({\bf 13 days})}
  \item \emph{Rule files} % TODO State tabs? State machines definition files?
    \subitem Define syntax.
    \subitem Parser function.
    \subitem Define files location by user.
\end{list}
% Now we have implemented one of the lacking features of the basic version with 
% more priority.
\begin{list}{-}{March 9th, 2012 ({\bf 9 days})}
  \item \emph{Control program with basic functionalities}
    \subitem Add transition.
    \subitem Manual event trigger.
%     \subitem Return all the states machines in a known graph syntax.
%       \subsubitem i.e. DOT.
%   \item Delivery this report.
\end{list}
With this step done, we have the first basic but functional version of the 
project.\\
% TODO This must be extended before the delivery of this report.
\begin{list}{-}{March 16th, 2012 ({\bf 7 days})}
  \item \emph{Shared library}\\
    This is expected to be easy and fast to finish, because it has been already implemented for the control program.
\end{list}
% TODO Do we have something to say here?
\begin{list}{-}{March 30th, 2012 ({\bf 14 days})}
  \item \emph{Propagate action}\\
    This was initially thought not to be an action, but to be an event filter with configuration files similar to the rule files.
\end{list}
% TODO Do we have something to say here?
\begin{list}{-}{April 13th, 2012 ({\bf 14 days})}
  \item \emph{remote events}\\
    Daemon socket ready to send and receive IP event messages.
  \item \emph{Plugin workers interface}
\end{list}
% TODO Do we have something to say here?
\begin{list}{-}{April 27th, 2012 ({\bf 14 days})}
  \item \emph{Test plugin}
    \subitem As much useful as time we have left to implement it.
  \item \emph{Current state storing}\\
    The idea is to let the user mark states which if \emph{reactord} is stopped, it will save that it was in this state before. When 
    \emph{reactord} is started again, it will begin the state machine at this state instead of the initial state.
    \subitem Define syntax.
    \subitem Parser function.
    \subitem Saver function.
    \subitem Define files location by user.
    \subitem Add the option to the control program.
  \item \emph{Trigger events on action finalization}\\
    We want a kind of event detected by \emph{reactord} itself, because its \emph{reactord} the software that generates it.
    This is the finalization of an action, so we can make a state machine wait until an action terminated. The events would be 
    finalize, error and success.
  \item \emph{Performance analysis}\\
    This is the last task because we want to test the whole software.
\end{list}
\begin{list}{-}{{\bf May 30th, 2012 ( 33 days)}}
  \item \emph{Write this document}
\end{list}
So the final deadline of the project is 05-30-2012, and the total number of days of this final project is 201 days. We have to take into
account that between these days there are holidays.
% TODO Extended list of tasks from the report, with a description for each task.
% \section{Temporal distribution}
% TODO Gantt diagram with explained timings.
\section{Final deviation}
The final deviation from the initial planning is not very relevant, even though some tasks hasn't been finally performed. That was
actually expected, and that's why in the last deadline we put the less important tasks.\\
Now we are going to list the deviations. We have to take into account that every time we change any deadline, the following deadlines are
altered as well because they need the same time as before. We are not going to explain all the deadlines modifications, only the ones
that needed a different amount of time than expected.\\
\begin{list}{-}{The deviations are explained as follows:}
  \item {\bf February 2nd, 2012} -> February 9th, 2011 (7 days more)\\
    I had never used \emph{Autotools} before, and in the beginning its usage seems a little tricky. The problem also was finding the proper
    structure of the source files.
  \item {\bf February 16th, 2012} -> March 1th, 2011 (7 days more)\\
    The problem here was a bad and hard implementation of the protocol between the daemon and the \emph{reactorctl}. The right
    reimplementation took its time.
  \item {\bf March 9th, 2012} -> March 20th, 2012 (3 days less)\\
    Once the protocol was defined, the control program was really easy to develop. All the hard work was already in the daemon.
  \item {\bf March 16th, 2012} -> March 23th, 2012 (4 days less)\\
    Easier to make than expected. Once we learned the usage of \emph{Libtool} the work was just moving code and changing includes.
  \item {\bf April 27th, 2012} -> April 27th, 2012 (7 days less)\\
    This is the deadline for the implementation process, so we did not change it. But already were a week behind the schedule, so there
    were things that couldn't be done. Finally we only made the first point, a simple plugin implementation that can be used as an example.
\end{list}

\section{Budget}
Here we will deem the economic cost for the realization of this project in a real environment.
\subsection{Software}
In this section we will analyse the costs of the software needed to develop this project. As we are talking about software and its price
depends on its license and the use that we make of it, we will specify the the license. \emph{Table \ref{tab:soft}} shows these costs.
\begin{table}[h]
  \begin{center}
    \begin{tabular}{ l c | p{2.5cm} |}
      \cline{2-3}
      & \multicolumn{1}{|l|}{License} & Cost \\ \hline
      \multicolumn{1}{|l|}{GCC} & \multicolumn{1}{|c|}{GPLv3} & 0€\\ \hline
      \multicolumn{1}{|l|}{GNU Build System} & \multicolumn{1}{|c|}{GPLv3} & 0€  \\ \hline
      \multicolumn{1}{|l|}{GIT} & \multicolumn{1}{|c|}{GPLv2} & 0€ \\ \hline
      \multicolumn{1}{|l|}{KDevelop} & \multicolumn{1}{|c|}{GPLv2} & 0€ \\ \hline
      \multicolumn{1}{|l|}{Kile} & \multicolumn{1}{|c|}{GPLv2} & 0€ \\ \hline
      \multicolumn{1}{|l|}{Valgrind} & \multicolumn{1}{|c|}{GPLv2} & 0€ \\ \hline
      \multicolumn{1}{|l|}{GLib} & \multicolumn{1}{|c|}{LGPLv2} & 0€ \\ \hline
      \multicolumn{1}{|l|}{libevent} & \multicolumn{1}{|c|}{BSD} & 0€ \\ \hline
      \multicolumn{1}{|l|}{Check} & \multicolumn{1}{|c|}{LGPL} & 0€ \\ \hline
      \multicolumn{1}{|l|}{Ubuntu} & \multicolumn{1}{|p{2.5cm}|}{Mainly the GNU GPL and various other free software licenses} & 0€ \\
      \hline
      \hline
      \multicolumn{1}{|l}{ Total} & & {\bf 0€}\\
      \hline
    \end{tabular}
  \end{center}
  \caption{Software cost}
  \label{tab:soft}
\end{table}
This is an advantage that usually comes with FLOSS, its also payment free.
\subsection{Hardware}
In the \emph{table \ref{tab:hard}} we show the hardware costs of our project. As we can see we only have a laptop and the source code hosting server.
The server is a free service for open-source projects, and the laptop is a mid-class computer, enough for our goals.\\
The recovery field is a factor to multiply to the cost. To calculate it we make the inverse of the time in which the hardware cost will be
recovered multiplied by the time it will be used for this project.
\begin{table}[h]
  \begin{center}
    \begin{tabular}{ l c c c | c |}
      \cline{2-5}
      & \multicolumn{1}{|l|}{Units} & \multicolumn{1}{|c|}{Recovery} & \multicolumn{1}{|c|}{Cost} & Total \\ \hline
      \multicolumn{1}{|l|}{Dell XPS M1330} & \multicolumn{1}{|c|}{1} & \multicolumn{1}{|c|}{0.25} &\multicolumn{1}{|c|}{1.100€} & 275€ \\ \hline
      \multicolumn{1}{|l|}{Source code hosting (GitHub)} & \multicolumn{1}{|c|}{1} & \multicolumn{1}{|c|}{0} &\multicolumn{1}{|c|}{0€} & 0€\\
      \hline
      \hline
      \multicolumn{1}{|l}{Total} & & & & {\bf 275€}\\
      \hline
    \end{tabular}
  \end{center}
  \caption{Hardware cost}
  \label{tab:hard}
\end{table}
\subsection{Personal}
This is the cost of the work-power for this project. We separate it in three roles with different costs per hour. All of them were
performed by me. The total number of hours stated by the rules of the final project is 600.\\
The results are shown in \emph{table \ref{tab:personal}}.
\begin{table}[h]
  \begin{center}
    \begin{tabular}{ l c c | c |}
      \cline{2-4}
      & \multicolumn{1}{|l|}{Hours} & \multicolumn{1}{|c|}{Cost/hour} & Total \\ \hline
  %     \multicolumn{1}{|l|}{Chief project engineer} & \multicolumn{1}{|c|}{} & \multicolumn{1}{|c|}{30€} & \\ \hline
      \multicolumn{1}{|l|}{Architect} & \multicolumn{1}{|c|}{190} & \multicolumn{1}{|c|}{60€} & 11.400€ \\ \hline
      \multicolumn{1}{|l|}{Software analyst} & \multicolumn{1}{|c|}{160} & \multicolumn{1}{|c|}{50€} & 8.000€\\ \hline
      \multicolumn{1}{|l|}{Programmer} & \multicolumn{1}{|c|}{250} & \multicolumn{1}{|c|}{30€} & 7.500€\\
      \hline
      \hline
      \multicolumn{1}{|l}{Total} & & & {\bf 26.900€}\\
      \hline
    \end{tabular}
  \end{center}
  \caption{Personal cost}
  \label{tab:personal}
\end{table}