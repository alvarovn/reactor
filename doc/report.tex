\documentclass[a4paper,11pt]{scrartcl}
\usepackage[utf8x]{inputenc}
\usepackage{fullpage}

%opening
\title{Unix Systems Event Manager}
\author{Álvaro Villalba Navarro\\
  Director: Juan José Costa Prats
}

\begin{document}
\maketitle

\begin{abstract}
In this initial report we will explain the problem about abstract events 
scheduling, handling and reaction in Unix systems that this final project 
intends to solve and describe the chosen solution. As the nature of the final 
project is time-limited we won't cover the implementation of the whole solution 
but the most important functionalities, so the goals of this project will be 
stated explicitly.\\
Furthermore this document will state where we are inside the time-line of the 
project, which is the current approach to the solution and how much work has to 
be done until the end of the given time. In a nutshell, which are the goals 
that have been achieved and which need to be achieved yet. This will be done by
defining a planning divided into explained tasks with assigned deadlines, a
final statement about the current work being done and a forecast about the
future of the project.\\
The reason for this document to be written in English, which is certainly not 
my mother tongue, is the Free Software nature of the project. This decision is 
expected to help its diffusion.
\end{abstract}

\newpage
\tableofcontents
\newpage

\section{Planning}
Here we will show the planning we intend to follow in order to `have things 
done` in time. It's mainly a list of tasks to perform ordered from most to less 
priority and assigned deadlines to sets of those tasks (or iterations):\\*
\\
% TODO Already done tasks
\begin{list}{-}{February 16th, 2012}
  \item Main data structure of the daemon done and working.
  \item Wrappers to all the third party data structures (mainly glib).
    \subitem As it's expected to improve the efficiency of the software in the 
	      future, we want to wrapper all the declarations of the third 
	      party basic data structures we use, so when we re-implement them
	      we won't have to change any other non-related code.
  \item Daemon socket ready to receive events from other processes.
    \subitem The poll management will be performed by 'libevent'.
  \item Execute shell commands functionality.
  \item Dummy first version of control program.
    \subitem Its purpose is to send events, so we can test the daemon.
\end{list}
After this deadline what we have are the very most basic and principal 
functionalities of the project. This is, a daemon with some states machines 
defined in it that receives event messages for those states machines from a 
socket and executes shell commands if assigned to the transitions.
There's no way to read the states machines from anywhere yet so it has some test 
hard-coded state machines.\\*
\\
\begin{list}{-}{February 29th, 2012}
  \item Transition tabs. % TODO State tabs? State machines definition files?
    \subitem Define syntax.
    \subitem Parser function.
    \subitem Saver function.
    \subitem Define files location by user.
\end{list}
Now we have implemented one of the lacking features of the basic version with 
more priority.\\*
\\
\begin{list}{-}{March 9th, 2012}
  \item Control program with the following functionalities:
    \subitem Add transition.
    \subitem Generate event. % TODO Generate?
    \subitem Return all the states machines in a known graph syntax.
      \subsubitem f.e. DOT.
\end{list}
With this step done, we have the first basic but functional version of the 
project.\\*
% TODO This must be extended before the delivery of this report.
\\
\begin{list}{-}{March 16th, 2012}
  \item Shared library.
    \subitem This is expected to be easy and fast to finish, because it has 
	      been already implemented for the control program.\\
	      'syslog.h' and 'libudev.h' style.
  \item Internal events. % TODO Is this really the best place to put that task?
    \subitem Implement a port to manage internal daemon events.
\end{list}
% TODO Do we have something to say here?
\begin{list}{-}{March 30th, 2012}
  \item Propagation rules data structure.
% TODO This must be updated when the decision is taken.
    \subitem This must be an update of the main data structure to also check
	      propagation rules or instead, creating a new parallel data 
	      structure.
  \item Propagation rules files.
    \subitem Define syntax.
    \subitem Parser function.
    \subitem Saver function.
    \subitem Define files location by user.
\end{list}
% TODO Do we have something to say here?
\begin{list}{-}{April 13th, 2012}
  \item Daemon socket ready to receive TCP/IP event messages.
  \item Plugin schedulers interface.
    \subitem This is the main part of the project which will make use of 
	      internal events by now.
\end{list}
% TODO Do we have something to say here?
\begin{list}{-}{April 27th, 2012}
  \item Test plugin.
    \subitem As much useful as time we have left to implement it.
  \item Current state storing.
    \subitem Define syntax.
    \subitem Parser function.
    \subitem Saver function.
    \subitem Define files location by user.
    \subitem Add the option to the 'transition tabs' syntax.
    \subitem Add the option to the control program.
  \item Action finalizing internal events.
    \subitem Finalize, error and success.
  \item Trigger currently valid events.
\end{list}
As we can see this set of tasks is a lot more dense than the previous. That's
because we are taking in account that maybe we can't develop all the features
because a lack of time, but again, maybe we don't have that lack of time and 
the project finishes with all those features in it. That's why the tasks are 
ordered by priority.\\
The minimum desired last task is the test plugin.


\end{document}
