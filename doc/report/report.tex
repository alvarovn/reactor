\documentclass[a4paper,11pt]{article}
\usepackage[utf8x]{inputenc}
\usepackage{fullpage}
\usepackage{graphicx}
\usepackage[small]{caption}
\usepackage{multicol}
% \usepackage{wrapfig}
\usepackage{listings}
\lstset{ 
  basicstyle=\footnotesize,  
}
%opening
\title{Unix Systems Event Manager \\ Initial report}
\author{Álvaro Villalba Navarro \\ Director: Juan José Costa Prats}

\begin{document}
\maketitle

\begin{abstract}
In this initial report we will explain the problems of general purpose job scheduling and
abstract events handling in Unix systems, and describe one solution implemented through this final project.
It's important to understand that this is not 
only an engineering final project, but also a personal project. As the nature of the final 
project is time-limited we won't cover the implementation of the whole solution but the most important 
functionalities, so the goals of this project will be stated explicitly.\\
Furthermore this document will state which are the goals that have been achieved and which need to be 
achieved yet. This will be done by defining a planning divided into tasks with assigned deadlines, a
final explanation about the work being done and a forecast about the future of the project.\\
The reason for this document to be written in English, which is not my native language, is that the project 
is under the free software terms. This decision is expected to help its diffusion and maybe catch the interest
of somebody.\\
% TODO Explain that it is not only a final project, but a personal project too.
\end{abstract}
% \newpage
% \tableofcontents
% \newpage
\section{Introduction}
In Unix-like operating systems there is a service which exists in almost every 
distribution or version and it's treated as an standard. This service is 
\emph{cron}, which according to Wikipedia:
\begin{quote}
  \emph{
    ``cron is a time-based job scheduler in Unix-like computer operating systems. cron 
    enables users to schedule jobs (commands or shell scripts) to run periodically at 
    certain times or dates. It is commonly used to automate system maintenance or 
    administration, though its general-purpose nature means that it can be used for other 
    purposes, such as connecting to the Internet and downloading email.''
  }
\end{quote}
It is a job scheduler\footnote{Should not be confused with process scheduling, which is 
the assignment of currently running processes to CPUs by the operating system.}, 
which means that it is a service in charge of running background 
executions. Job schedulers must be based on something to decide when is the right moment
to run a job. Most of job schedulers are workload-based, so they schedule their jobs by
taking into account the available resources, to run them when the computer is idle.\\
However, as we have seen on the Wikipedia article, cron is time-based, so it will run its 
jobs on defined instants.\\
There are more services with job-scheduling capabilities based on other kinds of events,
like \emph{udev} which is a device manager that can run shell commands when events such as a 'new
device connection' occur. Or \emph{syslog}, that can also run shell commands when it receives a
log message from an application.
\subsection{Problem}
This project intends to solve the problem of scheduling jobs using an abstract concept of \emph{event}, so we 
are not that much limited by the nature of the events that trigger the jobs, such as time or workload.\\
That alone is not a problem to solve as we have services for each kind of events.
\begin{list}{-}{But what we want is a service that:}
  \item Can have multiple events as a requirement for a job.
  \item Adds the ability of detecting new kinds of events to the system.
  \item Runs a state machine.
  \item Propagates events through IP.
\end{list}
The first point means that we want our service to notice various events of different kinds 
to execute a single job. It should also permit logical operations between those event notices
as condition to execute a job, like '\emph{if noticed event\_x or event\_y then run z}'.\\
The second point refers to the fact that as we said before, Unix-like systems have several
job schedulers for various kinds of events, generally each for one kind. But as we want to
be able to combine this kinds of events, it's also desirable to have more than we currently 
have. For example geolocation, system sensors, weather forecast or file monitoring events 
could be quite useful.\\
If we receive the event ``10:00AM'' and later the event ``11:00AM'', we can define the period of time 
``between 10AM and 11AM''. This period is a state, and we would like to execute jobs only in this state every 
time we receive some random event, but not when we receive it at 12AM. This is simply defining a 
\emph{state machine}, and it's explained in depth in the following section.\\
The last entry of the list means that a local service could propagate an event to the same service in another
computer in order to inform of that event and usually running a job there. 
\subsubsection{Example}
\begin{figure}
  \centering
    \includegraphics[height=16mm]{img/cronsm1}
  \caption{cron state machine example.}
  \label{fig:cronsm1}
\end{figure}
\begin{figure}
  \centering
    \includegraphics[height=40mm]{img/reactorsm1}
  \caption{Proposed service state machine example.}
  \label{fig:reactorsm1}
\end{figure}
Cron is a simple and known enough example of a job scheduler, so we will
use it to illustrate what we have and what we want. If we see every entry of a 
crontab as a state machine we would have something like shown on \emph{figure \ref{fig:cronsm1}}.
As we can see there is only one state (initial) and a single \emph{transition}. Over the transition we have an
event and a job separated by the character \emph{/}. In a state machine a job is called \emph{action}, so that is
the way we will call them from now on.\\
This is what cron can do, but what we actually would like to have on our service is something like 
\emph{figure \ref{fig:reactorsm1}}, where we can see multiple states, events of different kinds,
and transitions. Imagine for this example that we have a system administrator working in a office with his own 
laptop. This state machine detects when he is working by waiting for a geolocation event that sites
the device at the office, \emph{and} two time events\footnote{It could be in fact just one event as in 
\emph{figure \ref{fig:cronsm1}}, but this way is more comprehensive.} that define the beginning
of the working day. What if we start the service after the time the event would be received? A solution 
is discussed in section \ref{sec:ctl}.\\
The first action is a script that would change the current KDE\footnote{KDE is a popular software compilation for 
Linux operating systems, which include a desktop environment.} Activity\footnote{A KDE Activity is a desktop 
environment configuration, which includes the wallpaper, desktop widgets and applications running.} to one 
configured for work, with a sober and relaxing wallpaper, and the applications needed to do his job.\\
Once he is working, he will be notified every time the server he is in charge of logs an error message. When he
accept the notification it will open the working directory of the server. As this event has been generated on the 
server, this is a remote event received through IP.\\
To detect the end of our working day and close the 'work' Activity the service must detect that he
is no more in a working day (time event) \emph{or} that he is out of the office (geolocation).\\
A remark about this state machine is that it has a final state, so it won't start again until the service is 
restarted. This behaviour can be easily changed to the cron's behaviour by removing the final state and sending the 
last two transitions to the initial state.
% \lstinputlisting[language=sh,label=list:activity1,caption={work\_kde\_activity.sh}]{code/work_kde_activity.sh}
% \lstinputlisting[language=sh,label=list:notify1,caption={notify.sh}]{code/notify.sh}
% \lstinputlisting[language=sh,label=list:sftp1,caption={server\_sftp.sh}]{code/server_sftp.sh}
% \lstinputlisting[language=sh,label=list:activity2,caption={stop\_work\_activity.sh}]{code/stop_work_activity.sh}
\subsection{Solution}
\begin{figure}
  \centering
    \includegraphics[height=100mm]{img/reactordiagram}
  \caption{Diagram of reactor.}
  \label{fig:reactordia1}
\end{figure}
In \emph{figure \ref{fig:reactordia1}} we have a sketch of the solution we propose to the previously explained 
problem. The project is, by now, globally called \emph{reactor}. reactor is formed by four principal components:
\subsubsection{Daemon}
\emph{reactord} in \emph{figure \ref{fig:reactordia1}}, is the main service that would receive events from the 
\emph{event sources}. It also has the state machine structure specified by some editable files
\tiny(\emph{state tabs}) \normalsize similar to crontabs or the udev rules files.
Transitions can also be added by the user during the runtime of the service using the command line 
program \tiny(\emph{section \ref{sec:ctl}}) \normalsize. 
This state machine is defined by transitions with a list of event notices identifications, an action, 
and '\emph{from}' and '\emph{to}' state identifications. If 'from' identification is empty, then it's the initial 
state. If 'to' identification is not the 'from' identification of any other transition, then it is a final state.\\
reactord could also act as a \emph{remote event source} sending events through a network to a remote reactord. 
Those events are the events received by local sources that are stated on the \emph{propagation rules files} 
to be resend to an IP address. We could have a system to send propagation rules of the remote events we are 
interested to the remote reactord, although this won't be part of the final project.
\subsubsection{Workers}
\emph{Workers}\footnote{A worker is a child process that runs in parallel with its father doing some specific job.} 
or \emph{event triggers} in the sketch, are one kind of local event sources. Their goal is to generate events of 
kinds which we don't have any job scheduler, so reactord must provide a good interface to make them easy to 
develop.\\
For doing that they will act as an external job scheduler would, with its \emph{event rules}
files to know when to trigger an event, but with the advantage of being completely integrated with reactord.\\
We mentioned in the example the problem that if an event that defines the beginning of a state happens before the 
service started, we lose that information and reactord won't know we actually aren't in the initial state. 
The daemon workers gives us a solution: We send to the workers the events we are currently waiting for and which 
the workers are in charge. The workers will check if the last time those events happened was after the last time
next states leaving events happened. If that's true those events are triggered and we enter the correct state.
That's something other event sources can't do, because they are not integrated with the daemon.
\subsubsection{Command line program}
\label{sec:ctl}
\emph{reactorctl} in the sketch, is the typical administration or control application for the daemon. It has two 
differenced functionalities: sending administration commands and manually triggering events to the daemon.\\
The administration commands would be adding transitions to the state machine without saving them on disk, ask 
the workers to trigger the past events to enter in valid current states and get the state machines in a 
interpretable format\footnote{Those formats could be DOT, GML or TGF.}.\\
The manually triggering events functionality is useful not only for debugging purpose, but also because it gives 
the ability to use existent job schedulers for sending events to reactord, so we don't have to reinvent the wheel.
Far from being part of the final project, it could be able to set the rules of well known job schedulers to send 
events to reactord without having to edit manually the rules files of each one.
\subsubsection{Library}
In the diagram we can see one last component which is a random application sending events to reactord. It 
should be done by sharing the main reactor functions in a shared library, giving access to any application 
to easily sending events or adding transitions. This is intended to help to reduce the number of daemons on 
the system loaded by applications that are just waiting for something to occur to show simple alerts or 
alarms when the application is closed.

% TODO Explain why Unix-like/Linux OS.
  
\section{Planning}
That is the planning to follow in order to finish the project in time. 
Here is also where the goals of the final project are clearly 
stated. It's mainly a list of tasks to perform ordered from most to less 
priority, and assigned deadlines to sets of those tasks:
% TODO Already done tasks
\small
\begin{multicols}{2}
\ \\
\begin{list}{-}{January 26th, 2012}
  \item Design of the project.
\end{list}
\begin{list}{-}{February 2nd, 2012}
  \item Makefile.am and configure.ac.
  \item Log functions.
  \item reactor users check.
\end{list}
\begin{list}{-}{February 16th, 2012}
  \item Main data structure of the daemon done and working.
%   \item Wrappers to all the third party data structures (mainly glib).
%     \subitem As it's expected to improve the efficiency of the software in the 
% 	      future, we want to wrap all the declarations of the third 
% 	      party basic data structures we use, so when we re-implement them
% 	      the changes to any other non-related code will be minimal.
  \item Daemon socket ready to receive events from other processes.
%     \subitem The poll management will be performed by 'libevent'.
  \item Execute shell commands functionality.
  \item Dummy first version of control program.
%     \subitem Its purpose is to send events, so we can test the daemon.
\end{list}
% TODO Move this.
% After this deadline what we have are the very most basic and principal 
% functionalities of the project. This is, a daemon with some states machines 
% defined in it that receives event messages for those states machines from a 
% socket and executes shell commands if assigned to the transitions.
% There's no way to read the states machines from anywhere yet so it has some test 
% hard-coded state machines.\\*
\begin{list}{-}{February 29th, 2012}
  \item \emph{State tabs}. % TODO State tabs? State machines definition files?
%     \subitem Define syntax.
%     \subitem Parser function.
%     \subitem Saver function.
%     \subitem Define files location by user.
\end{list}
% Now we have implemented one of the lacking features of the basic version with 
% more priority.
\begin{list}{-}{March 9th, 2012}
  \item \emph{Control program with basic functionalities}.
%     \subitem Add transition.
%     \subitem Manual event trigger.
%     \subitem Return all the states machines in a known graph syntax.
%       \subsubitem f.e. DOT.
  \item Delivery this report.
\end{list}
% With this step done, we have the first basic but functional version of the 
% project.
% TODO This must be extended before the delivery of this report.
\begin{list}{-}{March 16th, 2012}
  \item Shared library.
%     \subitem This is expected to be easy and fast to finish, because it has 
% 	      been already implemented for the control program.\\
% 	      'syslog.h' and 'libudev.h' style.
%   \item Internal events. % TODO Is this really the best place to put that task?
%     \subitem Implement a port to manage internal daemon events.
\end{list}
% TODO Do we have something to say here?
\begin{list}{-}{March 30th, 2012}
  \item Propagation rules data structure.
% TODO This must be updated when the decision is taken.
%     \subitem This must be an update of the main data structure to also check
% 	      propagation rules or instead, creating a new parallel data 
% 	      structure.
  \item Propagation rules files.
%     \subitem Define syntax.
%     \subitem Parser function.
%     \subitem Saver function.
%     \subitem Define files location by user.
\end{list}
% TODO Do we have something to say here?
\begin{list}{-}{April 13th, 2012}
  \item Daemon socket ready to send and receive IP event messages.
  \item Plugin workers interface.
%     \subitem This is the main part of the project which will make use of 
% 	      internal events by now.
\end{list}
% TODO Do we have something to say here?
\begin{list}{-}{April 27th, 2012}
  \item Test plugin.
%     \subitem As much useful as time we have left to implement it.
  \item Current state storing.
%     \subitem Define syntax.
%     \subitem Parser function.
%     \subitem Saver function.
%     \subitem Define files location by user.
%     \subitem Add the option to the 'transition tabs' syntax.
%     \subitem Add the option to the control program.
  \item Trigger events on action finalization.
%     \subitem Finalize, error and success.
  \item Trigger currently valid events.
\end{list}
% As we can see this set of tasks is a lot more dense than the previous. That's
% because we are taking in account that maybe we can't develop all the features
% because a lack of time, but again, maybe we don't have that lack of time and 
% the project finishes with all those features in it. That's why the tasks are 
% ordered by priority.\\
% The minimum desired last task is the test plugin.
\end{multicols}
\normalsize
Right now we are at at March 9th, 2012 deadline and we haven't achieved all the goals
that we anticipated, even though we have most of the tasks done. The task that needs
more work to be done and was foreseen to be already done is the state tabs task, that
includes a syntax definition (already done), files destination (already done), parser 
and builder.\\
reactorctl has the basic functionalities done but it lacks a good argument entry for the
user.\\
Recently we noticed a malfunction on the state machine structure, and the socket 
communication between reactorctl and reactord can be improved.\\
\\
Because of the experience in this project until now, we believe that this planning may be
altered. That's why we consider the last deadline tasks as optional and they will be done if 
there is time left.

\end{document}
